\section{Analisi dei requisiti}

% TODO
% Is there any implicit requirement hidden within this project's requirements?
% %
% Is there any implicit hypothesis hidden within this project's requirements?
% %
% Are there any non-functional requirements implied by this project's requirements?

% What model / paradigm / techonology is the best suited to face this project's requirements?
% %
% What's the abstraction gap among the available models / paradigms / techonologies and the problem to be solved?

La fase di analisi dei requisiti è stata fondamentale, in quanto ha permesso uno studio approfondito delle tecnologie da utilizzare per implementare al meglio il progetto.

\subsection{Il framework JADE}
Il framework di appoggio scelto per lo sviluppo del progetto è stato JADE: JAVA Agent DEvelopment framework. Il suoo utilizzo porta infatti con se molti vantaggi. In primis, la sua storia di sviluppo lo rende un framework molto stabile e maturo, utilizzato in contesti aziendali da diverso tempo e quindi adeguatamente testato. Questo permette inoltre l'utilizzo di funzionalità avanzate, che si era inizialmente pensato di utilizzare, quali la definizione di ontologie a livello di programma (TODO). Oltre a ciò, il framework era già stato approcciato durante il corso e il suo utilizzo ha permesso di focalizzare gli sforzi e gli studi sull'utilizzo di tecnologie non ancora conosciute.
Come ultimo punto, la documentazione di JADE è curata, il che agevola ulteriormente nello sviluppo di applicazioni di tale complessità (TODO).
