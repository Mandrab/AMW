\section{Analisi dei requisiti}
% Is there any implicit requirement hidden within this project's requirements? Is there any implicit hypothesis hidden within this project's requirements? Are there any non-functional requirements implied by this project's requirements?
% What model / paradigm / techonology is the best suited to face this project's requirements? What's the abstraction gap among the available models / paradigms / techonologies and the problem to be solved?

I requisiti di alto livello definiti nella sezione precedente ci fornisco un punto di partenza, il quale non risulta però sufficiente alla scelta tecnologica ed al design. In questa sezione saranno dunque analizzate proprietà implicite che possiamo derivare dalle precedenti definizioni, in modo da ottenere e chiarire le caratteristiche richieste al nostro sistema.

\parag
Successivamente, si analizzeranno alcune tecnologie e/o paradigmi che possono risultare utili al completamento dei requisiti.

\subsection{Scalabilità}
Man mano che la domanda aumenta, gli elementi del sistema dovrebbero poter essere aggiunti in modo da ampliare le capacità dello stesso. Ciò è tuttavia fattibile principalmente per le entità software, mentre quelle hardware richiederebbero per questo un design dedicato allo scopo. Non volendo entrare in dettagli tanto profondi, si considererà che il design adeguato per queste sia già fornito e si modelleranno semplicemente i comportamenti ad alto livello atti a raggiungere una scalabilità teorica.

\subsection{Resilienza}
Il malfunzionamento di un entità del sistema non dovrebbe bloccare il funzionamento dello stesso. Ci\'o \'e tuttavia fattibile solo fino ad un certo punto. Le entit\'a hardware del sistema sono infatti aggiungibili e rimpiazzabili in maniera limitata, a causa della componente fisica del sistema stesso. Limitatamente alle possibilit\'a disponibili, si cercher\'a di mantenere il sistema funzionante tramite distribuzione, indipendenza e aggiunta di nuove entit\'a al sistema stesso.

\subsection{Semplicità di messa in funzionamento}
Il requisito precedentemente definito cita la possibilit\'a di aggiungere nuove entit\'a al sistema. A tal scopo si ritiene necessario che, ad esempio nel caso di fallimento di un componente, questo possa essere facilmente rimpiazzato (quanto meno nel caso software).

% \subsection{Comunicazione}
% per scalabilità. black board vs message passing
% \subsection{Sistema distribuito}
% per scalabilità e resilienza. jason deve funzionare su jade

\subsection{Le tecnologie scelte}
Diversi framework forniscono differenti modalità di programmazione. A causa di ciò, un'oculata scelta degli stessi risulta fondamentale per un corretto sviluppo del sistema. Questi devono infatti favorire il raggiungimento dei requisiti precedentemente definiti.\\
Di seguito verranno descritte le tecnologie scelte e le caratteristiche che hanno portato alla loro adozione.

\subsubsection{Il framework JADE}
Il framework di appoggio scelto per lo sviluppo del progetto è stato JADE: JAVA Agent DEvelopment framework. Questo permette lo sviluppo di applicativi distribuiti basati sulle astrazioni degli agenti.

Il suo utilizzo porta molti vantaggi. In primis, la sua storia di sviluppo lo rende un framework molto stabile e maturo, utilizzato in contesti aziendali da diverso tempo e quindi adeguatamente testato ed affidabile. Questo include inoltre funzionalità avanzate, di cui si è valutato l'utilizzo, quali la definizione di ontologie a livello di programma.

Il framework è stato inoltre approcciato durante il corso e la sua adozione ha permesso di focalizzare gli sforzi e gli studi sull'utilizzo di tecnologie non ancora conosciute.

Come ultimo punto, la documentazione di JADE è curata, il che agevola ulteriormente nello sviluppo di applicazioni di notevole complessità.
% TODO maybe confronto con tucson e spiegazione scelta

\subsubsection{L'interprete Jason}
Come tecnologia per lo sviluppo di alcuni degli agenti del sistema si è scelto di utilizzare il linguaggio ASL (AgentSpeakLanguage) nella sua estensione interpretata da Jason. Seppur ciò non fosse necessario allo sviluppo del sistema, le ragioni che hanno spinto a questa scelta sono legate alla volontà di apprendere il modello di programmazione BDI (Belief, Desire, Intention).

L'adozione di questa tecnologia ha anche permesso di entrare a contatto sia con lo standard di comunicazione FIPA-ACL (negli agenti JADE) sia con quello KQML (negli agenti Jason), dando modo di valutare ed utilizzare entrambi.

\subsubsection{La piattaforma Docker}
La tecnologia Docker permette di isolare i componenti del sistema e di eseguirli in maniera indipendente. Il suo utilizzo porta numerosi vantaggi, quali una notevole semplicità di deployment e scalabilità, a partire da una semplice immagine di base.

Docker modularizza inoltre la singola entità del sistema, permettendo una compartimentalizzazione implicita e, in questo contesto, la possibilità di simulare una distribuzione su diverse macchine. Le istanze delle immagini docker comunicano infatti su di una rete fornita dalla piattaforma, simulando quindi implicitamente una distribuzione del sistema.

Come ultimo punto, l'utilizzo di immagini dockerizzate permette di aggiungere in maniera semplice nuovi agenti al sistema durante le simulazioni.

\parag
L'utilizzo di questa tecnologia non era ovviamente necessario allo scopo base del progetto, ma la possibilità di simulare una distribuzione delle entità in rete era auspicabile e se ne è perciò deciso l'uso.

\subsubsection{Il tool di automazione Gradle}
Gradle è un tool di build automatica solitamente utilizzato per applicativi Java o funzionanti su JVM. Nel progetto è stato utilizzato per la definizione di task/comandi `user friendly' per la messa in esecuzione del sistema in tutte le sue componenti. Esso viene principalmente utilizzato per la creazione e l'avvio dei componenti dockerizzati, ma si occupa anche dell'esecuzione del client e del container principale di JADE. Quest'ultimo è infatti avviato direttamente sulla macchina, che funziona da main host per l'infrastruttura (e non è quindi dockerizzato).

L'utilizzo di Gradle permette infine un avvio istantaneo di tutte le componenti del sistema multi-agente tramite l'utilizzo di uno o pochi comandi.
