\documentclass{scrartcl}
\usepackage[utf8]{inputenc}
\usepackage{hyperref}
\usepackage{url}
\usepackage{natbib}
\usepackage{graphicx}
\usepackage{listings}
\usepackage[dvipsnames]{xcolor}

\newcommand{\emailaddr}[1]{\href{mailto:#1}{\texttt{#1}}}

\newcommand{\parag}{\vspace{6pt}\par}

\lstdefinelanguage{Kotlin}{
    basicstyle=\ttfamily\small,
    comment=[l]{//},
    commentstyle={\color{gray}\ttfamily},
    morecomment=[s]{/*}{*/},
    morestring=[b]",
    morestring=[s]{"""*}{*"""},
    morecomment=[s][\color{RedOrange}]{@}{\ }, % annotations
    emph={filter, first, firstOrNull, forEach, lazy, map, mapNotNull, println},
    emphstyle={\color{OrangeRed}},
    identifierstyle=\color{black},
    keywords={!in, !is, abstract, actual, annotation, as, as?, break, by, catch, class, companion, const, constructor, continue, crossinline, data, delegate, do, dynamic, else, enum, expect, external, false, field, file, final, finally, for, fun, get, if, import, in, infix, init, inline, inner, interface, internal, is, lateinit, noinline, null, object, open, operator, out, override, package, param, private, property, protected, public, receiveris, reified, return, return@, sealed, set, setparam, super, suspend, tailrec, this, throw, true, try, typealias, typeof, val, var, vararg, when, where, while},
    keywordstyle={\color{NavyBlue}\bfseries},
    ndkeywords={REQUEST, CONFIRM, INFORM, agent, commandManager, orderManager, robotPicker, warehouseMapper},
    ndkeywordstyle={\color{purple}},
    sensitive=true,
    stringstyle={\color{ForestGreen}\ttfamily},
}

\newlength\Colsep
\setlength\Colsep{10pt}

\title{\LARGE
    AMW - Agents Managed Warehouse
}

\author{
    Paolo Baldini \\ \emailaddr{paolo.baldini6@studio.unibo.it}
}

\begin{document}

\maketitle

%%%%%%%%%%%%%%%%%%%%%%%%%%%%%%%%%%%%%%%%%%%%%%%%%%%%%%%%%%%%%%%%%%%%%%%%%%%%%%%
%   ABSTRACT
%%%%%%%%%%%%%%%%%%%%%%%%%%%%%%%%%%%%%%%%%%%%%%%%%%%%%%%%%%%%%%%%%%%%%%%%%%%%%%%

\begin{abstract}
% TODO
% In the last years warehouses are getting more and more complicated. The increasing need of efficiency has pushed companies to provide their workers with tools that make jobs easier and faster. The largest ones have even substitute part of the human work with automatic machines. In this latter case, the most common approaches try to automatize through use of few PLC that manages the whole system.
% \parag
% Through the implementation of a system in which every machine appear as a robotic agent, I want to give a distributed alternative to the actual centralization practice. In this view, machines should manage incoming orders, retrieve required products and be as flexible and reusable as possible through the presence of an automatic update of \textit{procedural knowledge}.

Rispetto a qualche decennio fa', oggi i magazzini hanno al loro interno una notevole mole di complessità. La ricerca di sempre più efficienza ha infatti portato col tempo le aziende a fornire ai loro dipendenti strumentazioni per rendere più agevole e produttivo il lavoro e, le più grandi di queste, hanno addirittura sostituito parte del lavoro dipendente con macchine totalmente automatizzate. Prendendo in considerazione quest’ultima casistica, la pratica attuale più comune consiste nell’automatizzare il deposito per mezzo di pochi PLC che si occupino della gestione del sistema nel suo complesso.
\parag
Tramite l'implementazione di un sistema in cui ogni macchina appaia come un agente robotico, si vuole fornire un’alternativa distribuita alla pratica di centralizzazione attuale. In questa visione, le macchine dovranno gestire ordini entranti, recuperare i prodotti richiesti e risultare il più flessibili possibile, anche tramite l'upgrade automatico di conoscenza procedurale, nello
svolgimento di attività ”non standard”.
\end{abstract}
\newpage

\tableofcontents
\newpage

%%%%%%%%%%%%%%%%%%%%%%%%%%%%%%%%%%%%%%%%%%%%%%%%%%%%%%%%%%%%%%%%%%%%%%%%%%%%%%%
%   SECTIONS
%%%%%%%%%%%%%%%%%%%%%%%%%%%%%%%%%%%%%%%%%%%%%%%%%%%%%%%%%%%%%%%%%%%%%%%%%%%%%%%

\section{Obiettivi e requisiti}
Il progetto mira allo sviluppo di un sistema (simulato) di controllo di un magazzino, il quale sarà gestito da agenti software e robotici. La scelta dell’implementazione simulata del sistema è dovuta alla mancanza di controllori fisici atti a rappresentare gli agenti robotici. Oltre a questo, la scelta permette di mettere in secondo piano le complicazioni
hardware, che non rappresentano invero il focus del progetto.
\parag
Nel sistema si prevederà la possibilità di sottoporre un ordine. I prodotti che lo compongono saranno quindi raggruppati dagli agenti robotici incaricati. La suddivisione dei compiti di recupero delle parti dell'ordine sarà diretta invece da un agente `coordinatore'.\newline
Oltre al `normale' funzionamento di un magazzino, si desidera inoltre permettere agli agenti di eseguire task non precedentemente conosciuti. Lo sviluppo di questa caratteristica si delinea nella creazione di un repository di piani (i.e. abilità) a cui i vari agenti possano collegarsi per ottenere la capacità richiesta.
\parag
A livello tecnologico, il progetto prevede lo sviluppo di agenti JADE e Jason (questi ultimi, programmati quindi nella versione estesa di AgentSpeak(L)). Gli agenti funzioneranno su di un infrastruttura distribuita JADE.
Allo scopo di sviluppare la capacità di scaricamento dei piani, si ritiene infine utile la definizione di ontologie atte a descrivere il dominio del magazzino.
\parag
Come obiettivi di apprendimento, ci si pongono quelli di acquisire conoscenza relativamente al modello BDI\footnote{Beliefs, Desires, Intentions}, approfondire aspetti più avanzati di JADE e per quanto
possibile, valutare l'utilizzo e l'impatto della rappresentazione di un dominio mediante ontologie in un sistema distribuito. Si valuta infine la possibilità di apprendimento di docker per il deployment delle componenti del sistema.
\parag
Un elenco formale dei requirements del progetto sarà fornito di seguito. TODO

\subsection{Requisiti funzionali}
\begin{itemize}
    \item Implementare un sistema dinamico in grado di permettere l'aggiunta di agenti runtime;
    \item Permettere l'interazione col sistema tramite un client dedicato;
    \item Rendere possibile l'esecuzione di task non precedentemente definiti;
    \item Rendere possibile la sottomissione di ordini da utenti umani esterni al sistema.
\end{itemize}

\subsection{Requisiti non funzionali}
\begin{itemize}
    \item Studiare Jason e il modello BDI ed utilizzarli per la creazione degli agenti del sistema `magazzino';
    \item Approfondire la conoscenza del framework JADE ed utilizzarlo come architettura (TODO architettura è corretto?) base per la comunicazione degli agenti distribuiti;
    \item Definire una ontologia rappresentante il magazzino da utilizzare per la definizione di azioni e conoscenze;
    \item Utilizzare un approccio test-driven per monitorare il corretto funzionamento del sistema. (TODO da fare)
\end{itemize}

\subsection{Risultati attesi}
Il risultato atteso è un sistema simulante un magazzino. Questo dovrà accettare ordini entranti e adoperarsi per il loro corretto completamento.\\
Si prevede la presenza di una interfaccia utente tramite la quale sia possibile sottomettere ordini, eseguire task e visualizzare lo stato del sistema. Quest'ultima caratteristica non sarà tuttavia considerata un elemento obbligatorio nella soluzione finale e ci si riserva la possibilità di cancellarne lo sviluppo in caso di problemi o se i tempi di sviluppo dovessero estendersi troppo (TODO).\\
Gli artefatti derivanti dallo sviluppo del progetto saranno immagini docker rappresentanti le entità del sistema ed un software client funzionante su JVM per l'interazione con esso.

% % TODO
% % - detailed description of the project goals, 
% % - requirements,
% % - expected outcomes.
% %
% % Use case Diagrams, examples, or Q/A simulations are welcome.

% L'obiettivo del progetto è lo sviluppo di un sistema di gestione di magazzini basato sull'utilizzo di macchine controllate da agenti autonomi (vedasi Figura \ref{fig:warehouse}). Questo dovrà risultare flessibile per quanto riguarda l'introduzione di nuove macchine e funzionalità e per quanto possibile resistente ai malfunzionamenti. % TODO è resistente? TODO consistenza?
% %
% \begin{figure}[h]\centering
%     \includegraphics[width=.8\textwidth]{section/goals/figure/warehouse.png}
%     \caption{Rappresentazione immaginaria del sistema. Sono visibili le scaffalature, i robot e i punti di raccolta merci.}
%     \label{fig:warehouse}
% \end{figure}
% \\
% Da un punto di vista funzionale, saranno presenti due possibili utenti del sistema: \textit{cliente} e \textit{amministratore}. Questi avranno accesso a differenti tipi di operazioni, le quali possono essere riassunte dal seguente elenco puntato e dal diagramma degli usi visibile in Figura \ref{fig:warehouse-use-case}.
% %
% \begin{itemize}
%     \item piazzare un ordine \textit{[cliente]}
%     \item controllare stato di un ordine \textit{[cliente]}
%     \item supervisionare lo stato del magazzino \textit{[admin]}
%     \item rifornire il magazzino \textit{[admin]}
%     \item sottomettere un task alle macchine del sistema \textit{[admin]}
% \end{itemize}
% %
% \begin{figure}[h]\centering
%     \includegraphics[width=.8\textwidth]{section/goals/figure/warehouse-use_case.png}
%     \caption{Principali azioni effettuabili in base alla tipologia di utente.}
%     \label{fig:warehouse-use-case}
% \end{figure}

% \subsection{Scenarios}
% % TODO
% % Informal description of the ways users are expected to interact with your project.
% % It should describe \emph{how} and \emph{why} a user should use / interact with the system.

% Se i principali 

% \subsection{Self-assessment policy}

% \begin{itemize}
%     \item How should the \emph{quality} of the \emph{produced software} be assessed?
    
%     \item How should the \emph{effectiveness} of the project outcomes be assessed?
% \end{itemize}

\section{Obiettivi e requisiti}
Il progetto mira allo sviluppo di un sistema (simulato) di controllo di un magazzino, il quale sarà gestito da agenti software e robotici. La scelta dell’implementazione simulata del sistema è dovuta alla mancanza di controllori fisici atti a rappresentare gli agenti robotici. Oltre a questo, la scelta permette di mettere in secondo piano le complicazioni
hardware, che non rappresentano invero il focus del progetto.
\parag
Nel sistema si prevederà la possibilità di sottoporre un ordine. I prodotti che lo compongono saranno quindi raggruppati dagli agenti robotici incaricati. La suddivisione dei compiti di recupero delle parti dell'ordine sarà diretta invece da un agente `coordinatore'.\newline
Oltre al `normale' funzionamento di un magazzino, si desidera inoltre permettere agli agenti di eseguire task non precedentemente conosciuti. Lo sviluppo di questa caratteristica si delinea nella creazione di un repository di piani (i.e. abilità) a cui i vari agenti possano collegarsi per ottenere la capacità richiesta.
\parag
A livello tecnologico, il progetto prevede lo sviluppo di agenti JADE e Jason (questi ultimi, programmati quindi nella versione estesa di AgentSpeak(L)). Gli agenti funzioneranno su di un infrastruttura distribuita JADE.
Allo scopo di sviluppare la capacità di scaricamento dei piani, si ritiene infine utile la definizione di ontologie atte a descrivere il dominio del magazzino.
\parag
Come obiettivi di apprendimento, ci si pongono quelli di acquisire conoscenza relativamente al modello BDI\footnote{Beliefs, Desires, Intentions}, approfondire aspetti più avanzati di JADE e per quanto
possibile, valutare l'utilizzo e l'impatto della rappresentazione di un dominio mediante ontologie in un sistema distribuito. Si valuta infine la possibilità di apprendimento di docker per il deployment delle componenti del sistema.
\parag
Un elenco formale dei requirements del progetto sarà fornito di seguito. TODO

\subsection{Requisiti funzionali}
\begin{itemize}
    \item Implementare un sistema dinamico in grado di permettere l'aggiunta di agenti runtime;
    \item Permettere l'interazione col sistema tramite un client dedicato;
    \item Rendere possibile l'esecuzione di task non precedentemente definiti;
    \item Rendere possibile la sottomissione di ordini da utenti umani esterni al sistema.
\end{itemize}

\subsection{Requisiti non funzionali}
\begin{itemize}
    \item Studiare Jason e il modello BDI ed utilizzarli per la creazione degli agenti del sistema `magazzino';
    \item Approfondire la conoscenza del framework JADE ed utilizzarlo come architettura (TODO architettura è corretto?) base per la comunicazione degli agenti distribuiti;
    \item Definire una ontologia rappresentante il magazzino da utilizzare per la definizione di azioni e conoscenze;
    \item Utilizzare un approccio test-driven per monitorare il corretto funzionamento del sistema. (TODO da fare)
\end{itemize}

\subsection{Risultati attesi}
Il risultato atteso è un sistema simulante un magazzino. Questo dovrà accettare ordini entranti e adoperarsi per il loro corretto completamento.\\
Si prevede la presenza di una interfaccia utente tramite la quale sia possibile sottomettere ordini, eseguire task e visualizzare lo stato del sistema. Quest'ultima caratteristica non sarà tuttavia considerata un elemento obbligatorio nella soluzione finale e ci si riserva la possibilità di cancellarne lo sviluppo in caso di problemi o se i tempi di sviluppo dovessero estendersi troppo (TODO).\\
Gli artefatti derivanti dallo sviluppo del progetto saranno immagini docker rappresentanti le entità del sistema ed un software client funzionante su JVM per l'interazione con esso.

% % TODO
% % - detailed description of the project goals, 
% % - requirements,
% % - expected outcomes.
% %
% % Use case Diagrams, examples, or Q/A simulations are welcome.

% L'obiettivo del progetto è lo sviluppo di un sistema di gestione di magazzini basato sull'utilizzo di macchine controllate da agenti autonomi (vedasi Figura \ref{fig:warehouse}). Questo dovrà risultare flessibile per quanto riguarda l'introduzione di nuove macchine e funzionalità e per quanto possibile resistente ai malfunzionamenti. % TODO è resistente? TODO consistenza?
% %
% \begin{figure}[h]\centering
%     \includegraphics[width=.8\textwidth]{section/goals/figure/warehouse.png}
%     \caption{Rappresentazione immaginaria del sistema. Sono visibili le scaffalature, i robot e i punti di raccolta merci.}
%     \label{fig:warehouse}
% \end{figure}
% \\
% Da un punto di vista funzionale, saranno presenti due possibili utenti del sistema: \textit{cliente} e \textit{amministratore}. Questi avranno accesso a differenti tipi di operazioni, le quali possono essere riassunte dal seguente elenco puntato e dal diagramma degli usi visibile in Figura \ref{fig:warehouse-use-case}.
% %
% \begin{itemize}
%     \item piazzare un ordine \textit{[cliente]}
%     \item controllare stato di un ordine \textit{[cliente]}
%     \item supervisionare lo stato del magazzino \textit{[admin]}
%     \item rifornire il magazzino \textit{[admin]}
%     \item sottomettere un task alle macchine del sistema \textit{[admin]}
% \end{itemize}
% %
% \begin{figure}[h]\centering
%     \includegraphics[width=.8\textwidth]{section/goals/figure/warehouse-use_case.png}
%     \caption{Principali azioni effettuabili in base alla tipologia di utente.}
%     \label{fig:warehouse-use-case}
% \end{figure}

% \subsection{Scenarios}
% % TODO
% % Informal description of the ways users are expected to interact with your project.
% % It should describe \emph{how} and \emph{why} a user should use / interact with the system.

% Se i principali 

% \subsection{Self-assessment policy}

% \begin{itemize}
%     \item How should the \emph{quality} of the \emph{produced software} be assessed?
    
%     \item How should the \emph{effectiveness} of the project outcomes be assessed?
% \end{itemize}

\section{Obiettivi e requisiti}
Il progetto mira allo sviluppo di un sistema (simulato) di controllo di un magazzino, il quale sarà gestito da agenti software e robotici. La scelta dell’implementazione simulata del sistema è dovuta alla mancanza di controllori fisici atti a rappresentare gli agenti robotici. Oltre a questo, la scelta permette di mettere in secondo piano le complicazioni
hardware, che non rappresentano invero il focus del progetto.
\parag
Nel sistema si prevederà la possibilità di sottoporre un ordine. I prodotti che lo compongono saranno quindi raggruppati dagli agenti robotici incaricati. La suddivisione dei compiti di recupero delle parti dell'ordine sarà diretta invece da un agente `coordinatore'.\newline
Oltre al `normale' funzionamento di un magazzino, si desidera inoltre permettere agli agenti di eseguire task non precedentemente conosciuti. Lo sviluppo di questa caratteristica si delinea nella creazione di un repository di piani (i.e. abilità) a cui i vari agenti possano collegarsi per ottenere la capacità richiesta.
\parag
A livello tecnologico, il progetto prevede lo sviluppo di agenti JADE e Jason (questi ultimi, programmati quindi nella versione estesa di AgentSpeak(L)). Gli agenti funzioneranno su di un infrastruttura distribuita JADE.
Allo scopo di sviluppare la capacità di scaricamento dei piani, si ritiene infine utile la definizione di ontologie atte a descrivere il dominio del magazzino.
\parag
Come obiettivi di apprendimento, ci si pongono quelli di acquisire conoscenza relativamente al modello BDI\footnote{Beliefs, Desires, Intentions}, approfondire aspetti più avanzati di JADE e per quanto
possibile, valutare l'utilizzo e l'impatto della rappresentazione di un dominio mediante ontologie in un sistema distribuito. Si valuta infine la possibilità di apprendimento di docker per il deployment delle componenti del sistema.
\parag
Un elenco formale dei requirements del progetto sarà fornito di seguito. TODO

\subsection{Requisiti funzionali}
\begin{itemize}
    \item Implementare un sistema dinamico in grado di permettere l'aggiunta di agenti runtime;
    \item Permettere l'interazione col sistema tramite un client dedicato;
    \item Rendere possibile l'esecuzione di task non precedentemente definiti;
    \item Rendere possibile la sottomissione di ordini da utenti umani esterni al sistema.
\end{itemize}

\subsection{Requisiti non funzionali}
\begin{itemize}
    \item Studiare Jason e il modello BDI ed utilizzarli per la creazione degli agenti del sistema `magazzino';
    \item Approfondire la conoscenza del framework JADE ed utilizzarlo come architettura (TODO architettura è corretto?) base per la comunicazione degli agenti distribuiti;
    \item Definire una ontologia rappresentante il magazzino da utilizzare per la definizione di azioni e conoscenze;
    \item Utilizzare un approccio test-driven per monitorare il corretto funzionamento del sistema. (TODO da fare)
\end{itemize}

\subsection{Risultati attesi}
Il risultato atteso è un sistema simulante un magazzino. Questo dovrà accettare ordini entranti e adoperarsi per il loro corretto completamento.\\
Si prevede la presenza di una interfaccia utente tramite la quale sia possibile sottomettere ordini, eseguire task e visualizzare lo stato del sistema. Quest'ultima caratteristica non sarà tuttavia considerata un elemento obbligatorio nella soluzione finale e ci si riserva la possibilità di cancellarne lo sviluppo in caso di problemi o se i tempi di sviluppo dovessero estendersi troppo (TODO).\\
Gli artefatti derivanti dallo sviluppo del progetto saranno immagini docker rappresentanti le entità del sistema ed un software client funzionante su JVM per l'interazione con esso.

% % TODO
% % - detailed description of the project goals, 
% % - requirements,
% % - expected outcomes.
% %
% % Use case Diagrams, examples, or Q/A simulations are welcome.

% L'obiettivo del progetto è lo sviluppo di un sistema di gestione di magazzini basato sull'utilizzo di macchine controllate da agenti autonomi (vedasi Figura \ref{fig:warehouse}). Questo dovrà risultare flessibile per quanto riguarda l'introduzione di nuove macchine e funzionalità e per quanto possibile resistente ai malfunzionamenti. % TODO è resistente? TODO consistenza?
% %
% \begin{figure}[h]\centering
%     \includegraphics[width=.8\textwidth]{section/goals/figure/warehouse.png}
%     \caption{Rappresentazione immaginaria del sistema. Sono visibili le scaffalature, i robot e i punti di raccolta merci.}
%     \label{fig:warehouse}
% \end{figure}
% \\
% Da un punto di vista funzionale, saranno presenti due possibili utenti del sistema: \textit{cliente} e \textit{amministratore}. Questi avranno accesso a differenti tipi di operazioni, le quali possono essere riassunte dal seguente elenco puntato e dal diagramma degli usi visibile in Figura \ref{fig:warehouse-use-case}.
% %
% \begin{itemize}
%     \item piazzare un ordine \textit{[cliente]}
%     \item controllare stato di un ordine \textit{[cliente]}
%     \item supervisionare lo stato del magazzino \textit{[admin]}
%     \item rifornire il magazzino \textit{[admin]}
%     \item sottomettere un task alle macchine del sistema \textit{[admin]}
% \end{itemize}
% %
% \begin{figure}[h]\centering
%     \includegraphics[width=.8\textwidth]{section/goals/figure/warehouse-use_case.png}
%     \caption{Principali azioni effettuabili in base alla tipologia di utente.}
%     \label{fig:warehouse-use-case}
% \end{figure}

% \subsection{Scenarios}
% % TODO
% % Informal description of the ways users are expected to interact with your project.
% % It should describe \emph{how} and \emph{why} a user should use / interact with the system.

% Se i principali 

% \subsection{Self-assessment policy}

% \begin{itemize}
%     \item How should the \emph{quality} of the \emph{produced software} be assessed?
    
%     \item How should the \emph{effectiveness} of the project outcomes be assessed?
% \end{itemize}

\section{Obiettivi e requisiti}
Il progetto mira allo sviluppo di un sistema (simulato) di controllo di un magazzino, il quale sarà gestito da agenti software e robotici. La scelta dell’implementazione simulata del sistema è dovuta alla mancanza di controllori fisici atti a rappresentare gli agenti robotici. Oltre a questo, la scelta permette di mettere in secondo piano le complicazioni
hardware, che non rappresentano invero il focus del progetto.
\parag
Nel sistema si prevederà la possibilità di sottoporre un ordine. I prodotti che lo compongono saranno quindi raggruppati dagli agenti robotici incaricati. La suddivisione dei compiti di recupero delle parti dell'ordine sarà diretta invece da un agente `coordinatore'.\newline
Oltre al `normale' funzionamento di un magazzino, si desidera inoltre permettere agli agenti di eseguire task non precedentemente conosciuti. Lo sviluppo di questa caratteristica si delinea nella creazione di un repository di piani (i.e. abilità) a cui i vari agenti possano collegarsi per ottenere la capacità richiesta.
\parag
A livello tecnologico, il progetto prevede lo sviluppo di agenti JADE e Jason (questi ultimi, programmati quindi nella versione estesa di AgentSpeak(L)). Gli agenti funzioneranno su di un infrastruttura distribuita JADE.
Allo scopo di sviluppare la capacità di scaricamento dei piani, si ritiene infine utile la definizione di ontologie atte a descrivere il dominio del magazzino.
\parag
Come obiettivi di apprendimento, ci si pongono quelli di acquisire conoscenza relativamente al modello BDI\footnote{Beliefs, Desires, Intentions}, approfondire aspetti più avanzati di JADE e per quanto
possibile, valutare l'utilizzo e l'impatto della rappresentazione di un dominio mediante ontologie in un sistema distribuito. Si valuta infine la possibilità di apprendimento di docker per il deployment delle componenti del sistema.
\parag
Un elenco formale dei requirements del progetto sarà fornito di seguito. TODO

\subsection{Requisiti funzionali}
\begin{itemize}
    \item Implementare un sistema dinamico in grado di permettere l'aggiunta di agenti runtime;
    \item Permettere l'interazione col sistema tramite un client dedicato;
    \item Rendere possibile l'esecuzione di task non precedentemente definiti;
    \item Rendere possibile la sottomissione di ordini da utenti umani esterni al sistema.
\end{itemize}

\subsection{Requisiti non funzionali}
\begin{itemize}
    \item Studiare Jason e il modello BDI ed utilizzarli per la creazione degli agenti del sistema `magazzino';
    \item Approfondire la conoscenza del framework JADE ed utilizzarlo come architettura (TODO architettura è corretto?) base per la comunicazione degli agenti distribuiti;
    \item Definire una ontologia rappresentante il magazzino da utilizzare per la definizione di azioni e conoscenze;
    \item Utilizzare un approccio test-driven per monitorare il corretto funzionamento del sistema. (TODO da fare)
\end{itemize}

\subsection{Risultati attesi}
Il risultato atteso è un sistema simulante un magazzino. Questo dovrà accettare ordini entranti e adoperarsi per il loro corretto completamento.\\
Si prevede la presenza di una interfaccia utente tramite la quale sia possibile sottomettere ordini, eseguire task e visualizzare lo stato del sistema. Quest'ultima caratteristica non sarà tuttavia considerata un elemento obbligatorio nella soluzione finale e ci si riserva la possibilità di cancellarne lo sviluppo in caso di problemi o se i tempi di sviluppo dovessero estendersi troppo (TODO).\\
Gli artefatti derivanti dallo sviluppo del progetto saranno immagini docker rappresentanti le entità del sistema ed un software client funzionante su JVM per l'interazione con esso.

% % TODO
% % - detailed description of the project goals, 
% % - requirements,
% % - expected outcomes.
% %
% % Use case Diagrams, examples, or Q/A simulations are welcome.

% L'obiettivo del progetto è lo sviluppo di un sistema di gestione di magazzini basato sull'utilizzo di macchine controllate da agenti autonomi (vedasi Figura \ref{fig:warehouse}). Questo dovrà risultare flessibile per quanto riguarda l'introduzione di nuove macchine e funzionalità e per quanto possibile resistente ai malfunzionamenti. % TODO è resistente? TODO consistenza?
% %
% \begin{figure}[h]\centering
%     \includegraphics[width=.8\textwidth]{section/goals/figure/warehouse.png}
%     \caption{Rappresentazione immaginaria del sistema. Sono visibili le scaffalature, i robot e i punti di raccolta merci.}
%     \label{fig:warehouse}
% \end{figure}
% \\
% Da un punto di vista funzionale, saranno presenti due possibili utenti del sistema: \textit{cliente} e \textit{amministratore}. Questi avranno accesso a differenti tipi di operazioni, le quali possono essere riassunte dal seguente elenco puntato e dal diagramma degli usi visibile in Figura \ref{fig:warehouse-use-case}.
% %
% \begin{itemize}
%     \item piazzare un ordine \textit{[cliente]}
%     \item controllare stato di un ordine \textit{[cliente]}
%     \item supervisionare lo stato del magazzino \textit{[admin]}
%     \item rifornire il magazzino \textit{[admin]}
%     \item sottomettere un task alle macchine del sistema \textit{[admin]}
% \end{itemize}
% %
% \begin{figure}[h]\centering
%     \includegraphics[width=.8\textwidth]{section/goals/figure/warehouse-use_case.png}
%     \caption{Principali azioni effettuabili in base alla tipologia di utente.}
%     \label{fig:warehouse-use-case}
% \end{figure}

% \subsection{Scenarios}
% % TODO
% % Informal description of the ways users are expected to interact with your project.
% % It should describe \emph{how} and \emph{why} a user should use / interact with the system.

% Se i principali 

% \subsection{Self-assessment policy}

% \begin{itemize}
%     \item How should the \emph{quality} of the \emph{produced software} be assessed?
    
%     \item How should the \emph{effectiveness} of the project outcomes be assessed?
% \end{itemize}

\section{Obiettivi e requisiti}
Il progetto mira allo sviluppo di un sistema (simulato) di controllo di un magazzino, il quale sarà gestito da agenti software e robotici. La scelta dell’implementazione simulata del sistema è dovuta alla mancanza di controllori fisici atti a rappresentare gli agenti robotici. Oltre a questo, la scelta permette di mettere in secondo piano le complicazioni
hardware, che non rappresentano invero il focus del progetto.
\parag
Nel sistema si prevederà la possibilità di sottoporre un ordine. I prodotti che lo compongono saranno quindi raggruppati dagli agenti robotici incaricati. La suddivisione dei compiti di recupero delle parti dell'ordine sarà diretta invece da un agente `coordinatore'.\newline
Oltre al `normale' funzionamento di un magazzino, si desidera inoltre permettere agli agenti di eseguire task non precedentemente conosciuti. Lo sviluppo di questa caratteristica si delinea nella creazione di un repository di piani (i.e. abilità) a cui i vari agenti possano collegarsi per ottenere la capacità richiesta.
\parag
A livello tecnologico, il progetto prevede lo sviluppo di agenti JADE e Jason (questi ultimi, programmati quindi nella versione estesa di AgentSpeak(L)). Gli agenti funzioneranno su di un infrastruttura distribuita JADE.
Allo scopo di sviluppare la capacità di scaricamento dei piani, si ritiene infine utile la definizione di ontologie atte a descrivere il dominio del magazzino.
\parag
Come obiettivi di apprendimento, ci si pongono quelli di acquisire conoscenza relativamente al modello BDI\footnote{Beliefs, Desires, Intentions}, approfondire aspetti più avanzati di JADE e per quanto
possibile, valutare l'utilizzo e l'impatto della rappresentazione di un dominio mediante ontologie in un sistema distribuito. Si valuta infine la possibilità di apprendimento di docker per il deployment delle componenti del sistema.
\parag
Un elenco formale dei requirements del progetto sarà fornito di seguito. TODO

\subsection{Requisiti funzionali}
\begin{itemize}
    \item Implementare un sistema dinamico in grado di permettere l'aggiunta di agenti runtime;
    \item Permettere l'interazione col sistema tramite un client dedicato;
    \item Rendere possibile l'esecuzione di task non precedentemente definiti;
    \item Rendere possibile la sottomissione di ordini da utenti umani esterni al sistema.
\end{itemize}

\subsection{Requisiti non funzionali}
\begin{itemize}
    \item Studiare Jason e il modello BDI ed utilizzarli per la creazione degli agenti del sistema `magazzino';
    \item Approfondire la conoscenza del framework JADE ed utilizzarlo come architettura (TODO architettura è corretto?) base per la comunicazione degli agenti distribuiti;
    \item Definire una ontologia rappresentante il magazzino da utilizzare per la definizione di azioni e conoscenze;
    \item Utilizzare un approccio test-driven per monitorare il corretto funzionamento del sistema. (TODO da fare)
\end{itemize}

\subsection{Risultati attesi}
Il risultato atteso è un sistema simulante un magazzino. Questo dovrà accettare ordini entranti e adoperarsi per il loro corretto completamento.\\
Si prevede la presenza di una interfaccia utente tramite la quale sia possibile sottomettere ordini, eseguire task e visualizzare lo stato del sistema. Quest'ultima caratteristica non sarà tuttavia considerata un elemento obbligatorio nella soluzione finale e ci si riserva la possibilità di cancellarne lo sviluppo in caso di problemi o se i tempi di sviluppo dovessero estendersi troppo (TODO).\\
Gli artefatti derivanti dallo sviluppo del progetto saranno immagini docker rappresentanti le entità del sistema ed un software client funzionante su JVM per l'interazione con esso.

% % TODO
% % - detailed description of the project goals, 
% % - requirements,
% % - expected outcomes.
% %
% % Use case Diagrams, examples, or Q/A simulations are welcome.

% L'obiettivo del progetto è lo sviluppo di un sistema di gestione di magazzini basato sull'utilizzo di macchine controllate da agenti autonomi (vedasi Figura \ref{fig:warehouse}). Questo dovrà risultare flessibile per quanto riguarda l'introduzione di nuove macchine e funzionalità e per quanto possibile resistente ai malfunzionamenti. % TODO è resistente? TODO consistenza?
% %
% \begin{figure}[h]\centering
%     \includegraphics[width=.8\textwidth]{section/goals/figure/warehouse.png}
%     \caption{Rappresentazione immaginaria del sistema. Sono visibili le scaffalature, i robot e i punti di raccolta merci.}
%     \label{fig:warehouse}
% \end{figure}
% \\
% Da un punto di vista funzionale, saranno presenti due possibili utenti del sistema: \textit{cliente} e \textit{amministratore}. Questi avranno accesso a differenti tipi di operazioni, le quali possono essere riassunte dal seguente elenco puntato e dal diagramma degli usi visibile in Figura \ref{fig:warehouse-use-case}.
% %
% \begin{itemize}
%     \item piazzare un ordine \textit{[cliente]}
%     \item controllare stato di un ordine \textit{[cliente]}
%     \item supervisionare lo stato del magazzino \textit{[admin]}
%     \item rifornire il magazzino \textit{[admin]}
%     \item sottomettere un task alle macchine del sistema \textit{[admin]}
% \end{itemize}
% %
% \begin{figure}[h]\centering
%     \includegraphics[width=.8\textwidth]{section/goals/figure/warehouse-use_case.png}
%     \caption{Principali azioni effettuabili in base alla tipologia di utente.}
%     \label{fig:warehouse-use-case}
% \end{figure}

% \subsection{Scenarios}
% % TODO
% % Informal description of the ways users are expected to interact with your project.
% % It should describe \emph{how} and \emph{why} a user should use / interact with the system.

% Se i principali 

% \subsection{Self-assessment policy}

% \begin{itemize}
%     \item How should the \emph{quality} of the \emph{produced software} be assessed?
    
%     \item How should the \emph{effectiveness} of the project outcomes be assessed?
% \end{itemize}

\section{Obiettivi e requisiti}
Il progetto mira allo sviluppo di un sistema (simulato) di controllo di un magazzino, il quale sarà gestito da agenti software e robotici. La scelta dell’implementazione simulata del sistema è dovuta alla mancanza di controllori fisici atti a rappresentare gli agenti robotici. Oltre a questo, la scelta permette di mettere in secondo piano le complicazioni
hardware, che non rappresentano invero il focus del progetto.
\parag
Nel sistema si prevederà la possibilità di sottoporre un ordine. I prodotti che lo compongono saranno quindi raggruppati dagli agenti robotici incaricati. La suddivisione dei compiti di recupero delle parti dell'ordine sarà diretta invece da un agente `coordinatore'.\newline
Oltre al `normale' funzionamento di un magazzino, si desidera inoltre permettere agli agenti di eseguire task non precedentemente conosciuti. Lo sviluppo di questa caratteristica si delinea nella creazione di un repository di piani (i.e. abilità) a cui i vari agenti possano collegarsi per ottenere la capacità richiesta.
\parag
A livello tecnologico, il progetto prevede lo sviluppo di agenti JADE e Jason (questi ultimi, programmati quindi nella versione estesa di AgentSpeak(L)). Gli agenti funzioneranno su di un infrastruttura distribuita JADE.
Allo scopo di sviluppare la capacità di scaricamento dei piani, si ritiene infine utile la definizione di ontologie atte a descrivere il dominio del magazzino.
\parag
Come obiettivi di apprendimento, ci si pongono quelli di acquisire conoscenza relativamente al modello BDI\footnote{Beliefs, Desires, Intentions}, approfondire aspetti più avanzati di JADE e per quanto
possibile, valutare l'utilizzo e l'impatto della rappresentazione di un dominio mediante ontologie in un sistema distribuito. Si valuta infine la possibilità di apprendimento di docker per il deployment delle componenti del sistema.
\parag
Un elenco formale dei requirements del progetto sarà fornito di seguito. TODO

\subsection{Requisiti funzionali}
\begin{itemize}
    \item Implementare un sistema dinamico in grado di permettere l'aggiunta di agenti runtime;
    \item Permettere l'interazione col sistema tramite un client dedicato;
    \item Rendere possibile l'esecuzione di task non precedentemente definiti;
    \item Rendere possibile la sottomissione di ordini da utenti umani esterni al sistema.
\end{itemize}

\subsection{Requisiti non funzionali}
\begin{itemize}
    \item Studiare Jason e il modello BDI ed utilizzarli per la creazione degli agenti del sistema `magazzino';
    \item Approfondire la conoscenza del framework JADE ed utilizzarlo come architettura (TODO architettura è corretto?) base per la comunicazione degli agenti distribuiti;
    \item Definire una ontologia rappresentante il magazzino da utilizzare per la definizione di azioni e conoscenze;
    \item Utilizzare un approccio test-driven per monitorare il corretto funzionamento del sistema. (TODO da fare)
\end{itemize}

\subsection{Risultati attesi}
Il risultato atteso è un sistema simulante un magazzino. Questo dovrà accettare ordini entranti e adoperarsi per il loro corretto completamento.\\
Si prevede la presenza di una interfaccia utente tramite la quale sia possibile sottomettere ordini, eseguire task e visualizzare lo stato del sistema. Quest'ultima caratteristica non sarà tuttavia considerata un elemento obbligatorio nella soluzione finale e ci si riserva la possibilità di cancellarne lo sviluppo in caso di problemi o se i tempi di sviluppo dovessero estendersi troppo (TODO).\\
Gli artefatti derivanti dallo sviluppo del progetto saranno immagini docker rappresentanti le entità del sistema ed un software client funzionante su JVM per l'interazione con esso.

% % TODO
% % - detailed description of the project goals, 
% % - requirements,
% % - expected outcomes.
% %
% % Use case Diagrams, examples, or Q/A simulations are welcome.

% L'obiettivo del progetto è lo sviluppo di un sistema di gestione di magazzini basato sull'utilizzo di macchine controllate da agenti autonomi (vedasi Figura \ref{fig:warehouse}). Questo dovrà risultare flessibile per quanto riguarda l'introduzione di nuove macchine e funzionalità e per quanto possibile resistente ai malfunzionamenti. % TODO è resistente? TODO consistenza?
% %
% \begin{figure}[h]\centering
%     \includegraphics[width=.8\textwidth]{section/goals/figure/warehouse.png}
%     \caption{Rappresentazione immaginaria del sistema. Sono visibili le scaffalature, i robot e i punti di raccolta merci.}
%     \label{fig:warehouse}
% \end{figure}
% \\
% Da un punto di vista funzionale, saranno presenti due possibili utenti del sistema: \textit{cliente} e \textit{amministratore}. Questi avranno accesso a differenti tipi di operazioni, le quali possono essere riassunte dal seguente elenco puntato e dal diagramma degli usi visibile in Figura \ref{fig:warehouse-use-case}.
% %
% \begin{itemize}
%     \item piazzare un ordine \textit{[cliente]}
%     \item controllare stato di un ordine \textit{[cliente]}
%     \item supervisionare lo stato del magazzino \textit{[admin]}
%     \item rifornire il magazzino \textit{[admin]}
%     \item sottomettere un task alle macchine del sistema \textit{[admin]}
% \end{itemize}
% %
% \begin{figure}[h]\centering
%     \includegraphics[width=.8\textwidth]{section/goals/figure/warehouse-use_case.png}
%     \caption{Principali azioni effettuabili in base alla tipologia di utente.}
%     \label{fig:warehouse-use-case}
% \end{figure}

% \subsection{Scenarios}
% % TODO
% % Informal description of the ways users are expected to interact with your project.
% % It should describe \emph{how} and \emph{why} a user should use / interact with the system.

% Se i principali 

% \subsection{Self-assessment policy}

% \begin{itemize}
%     \item How should the \emph{quality} of the \emph{produced software} be assessed?
    
%     \item How should the \emph{effectiveness} of the project outcomes be assessed?
% \end{itemize}

\section{Obiettivi e requisiti}
Il progetto mira allo sviluppo di un sistema (simulato) di controllo di un magazzino, il quale sarà gestito da agenti software e robotici. La scelta dell’implementazione simulata del sistema è dovuta alla mancanza di controllori fisici atti a rappresentare gli agenti robotici. Oltre a questo, la scelta permette di mettere in secondo piano le complicazioni
hardware, che non rappresentano invero il focus del progetto.
\parag
Nel sistema si prevederà la possibilità di sottoporre un ordine. I prodotti che lo compongono saranno quindi raggruppati dagli agenti robotici incaricati. La suddivisione dei compiti di recupero delle parti dell'ordine sarà diretta invece da un agente `coordinatore'.\newline
Oltre al `normale' funzionamento di un magazzino, si desidera inoltre permettere agli agenti di eseguire task non precedentemente conosciuti. Lo sviluppo di questa caratteristica si delinea nella creazione di un repository di piani (i.e. abilità) a cui i vari agenti possano collegarsi per ottenere la capacità richiesta.
\parag
A livello tecnologico, il progetto prevede lo sviluppo di agenti JADE e Jason (questi ultimi, programmati quindi nella versione estesa di AgentSpeak(L)). Gli agenti funzioneranno su di un infrastruttura distribuita JADE.
Allo scopo di sviluppare la capacità di scaricamento dei piani, si ritiene infine utile la definizione di ontologie atte a descrivere il dominio del magazzino.
\parag
Come obiettivi di apprendimento, ci si pongono quelli di acquisire conoscenza relativamente al modello BDI\footnote{Beliefs, Desires, Intentions}, approfondire aspetti più avanzati di JADE e per quanto
possibile, valutare l'utilizzo e l'impatto della rappresentazione di un dominio mediante ontologie in un sistema distribuito. Si valuta infine la possibilità di apprendimento di docker per il deployment delle componenti del sistema.
\parag
Un elenco formale dei requirements del progetto sarà fornito di seguito. TODO

\subsection{Requisiti funzionali}
\begin{itemize}
    \item Implementare un sistema dinamico in grado di permettere l'aggiunta di agenti runtime;
    \item Permettere l'interazione col sistema tramite un client dedicato;
    \item Rendere possibile l'esecuzione di task non precedentemente definiti;
    \item Rendere possibile la sottomissione di ordini da utenti umani esterni al sistema.
\end{itemize}

\subsection{Requisiti non funzionali}
\begin{itemize}
    \item Studiare Jason e il modello BDI ed utilizzarli per la creazione degli agenti del sistema `magazzino';
    \item Approfondire la conoscenza del framework JADE ed utilizzarlo come architettura (TODO architettura è corretto?) base per la comunicazione degli agenti distribuiti;
    \item Definire una ontologia rappresentante il magazzino da utilizzare per la definizione di azioni e conoscenze;
    \item Utilizzare un approccio test-driven per monitorare il corretto funzionamento del sistema. (TODO da fare)
\end{itemize}

\subsection{Risultati attesi}
Il risultato atteso è un sistema simulante un magazzino. Questo dovrà accettare ordini entranti e adoperarsi per il loro corretto completamento.\\
Si prevede la presenza di una interfaccia utente tramite la quale sia possibile sottomettere ordini, eseguire task e visualizzare lo stato del sistema. Quest'ultima caratteristica non sarà tuttavia considerata un elemento obbligatorio nella soluzione finale e ci si riserva la possibilità di cancellarne lo sviluppo in caso di problemi o se i tempi di sviluppo dovessero estendersi troppo (TODO).\\
Gli artefatti derivanti dallo sviluppo del progetto saranno immagini docker rappresentanti le entità del sistema ed un software client funzionante su JVM per l'interazione con esso.

% % TODO
% % - detailed description of the project goals, 
% % - requirements,
% % - expected outcomes.
% %
% % Use case Diagrams, examples, or Q/A simulations are welcome.

% L'obiettivo del progetto è lo sviluppo di un sistema di gestione di magazzini basato sull'utilizzo di macchine controllate da agenti autonomi (vedasi Figura \ref{fig:warehouse}). Questo dovrà risultare flessibile per quanto riguarda l'introduzione di nuove macchine e funzionalità e per quanto possibile resistente ai malfunzionamenti. % TODO è resistente? TODO consistenza?
% %
% \begin{figure}[h]\centering
%     \includegraphics[width=.8\textwidth]{section/goals/figure/warehouse.png}
%     \caption{Rappresentazione immaginaria del sistema. Sono visibili le scaffalature, i robot e i punti di raccolta merci.}
%     \label{fig:warehouse}
% \end{figure}
% \\
% Da un punto di vista funzionale, saranno presenti due possibili utenti del sistema: \textit{cliente} e \textit{amministratore}. Questi avranno accesso a differenti tipi di operazioni, le quali possono essere riassunte dal seguente elenco puntato e dal diagramma degli usi visibile in Figura \ref{fig:warehouse-use-case}.
% %
% \begin{itemize}
%     \item piazzare un ordine \textit{[cliente]}
%     \item controllare stato di un ordine \textit{[cliente]}
%     \item supervisionare lo stato del magazzino \textit{[admin]}
%     \item rifornire il magazzino \textit{[admin]}
%     \item sottomettere un task alle macchine del sistema \textit{[admin]}
% \end{itemize}
% %
% \begin{figure}[h]\centering
%     \includegraphics[width=.8\textwidth]{section/goals/figure/warehouse-use_case.png}
%     \caption{Principali azioni effettuabili in base alla tipologia di utente.}
%     \label{fig:warehouse-use-case}
% \end{figure}

% \subsection{Scenarios}
% % TODO
% % Informal description of the ways users are expected to interact with your project.
% % It should describe \emph{how} and \emph{why} a user should use / interact with the system.

% Se i principali 

% \subsection{Self-assessment policy}

% \begin{itemize}
%     \item How should the \emph{quality} of the \emph{produced software} be assessed?
    
%     \item How should the \emph{effectiveness} of the project outcomes be assessed?
% \end{itemize}

\section{Obiettivi e requisiti}
Il progetto mira allo sviluppo di un sistema (simulato) di controllo di un magazzino, il quale sarà gestito da agenti software e robotici. La scelta dell’implementazione simulata del sistema è dovuta alla mancanza di controllori fisici atti a rappresentare gli agenti robotici. Oltre a questo, la scelta permette di mettere in secondo piano le complicazioni
hardware, che non rappresentano invero il focus del progetto.
\parag
Nel sistema si prevederà la possibilità di sottoporre un ordine. I prodotti che lo compongono saranno quindi raggruppati dagli agenti robotici incaricati. La suddivisione dei compiti di recupero delle parti dell'ordine sarà diretta invece da un agente `coordinatore'.\newline
Oltre al `normale' funzionamento di un magazzino, si desidera inoltre permettere agli agenti di eseguire task non precedentemente conosciuti. Lo sviluppo di questa caratteristica si delinea nella creazione di un repository di piani (i.e. abilità) a cui i vari agenti possano collegarsi per ottenere la capacità richiesta.
\parag
A livello tecnologico, il progetto prevede lo sviluppo di agenti JADE e Jason (questi ultimi, programmati quindi nella versione estesa di AgentSpeak(L)). Gli agenti funzioneranno su di un infrastruttura distribuita JADE.
Allo scopo di sviluppare la capacità di scaricamento dei piani, si ritiene infine utile la definizione di ontologie atte a descrivere il dominio del magazzino.
\parag
Come obiettivi di apprendimento, ci si pongono quelli di acquisire conoscenza relativamente al modello BDI\footnote{Beliefs, Desires, Intentions}, approfondire aspetti più avanzati di JADE e per quanto
possibile, valutare l'utilizzo e l'impatto della rappresentazione di un dominio mediante ontologie in un sistema distribuito. Si valuta infine la possibilità di apprendimento di docker per il deployment delle componenti del sistema.
\parag
Un elenco formale dei requirements del progetto sarà fornito di seguito. TODO

\subsection{Requisiti funzionali}
\begin{itemize}
    \item Implementare un sistema dinamico in grado di permettere l'aggiunta di agenti runtime;
    \item Permettere l'interazione col sistema tramite un client dedicato;
    \item Rendere possibile l'esecuzione di task non precedentemente definiti;
    \item Rendere possibile la sottomissione di ordini da utenti umani esterni al sistema.
\end{itemize}

\subsection{Requisiti non funzionali}
\begin{itemize}
    \item Studiare Jason e il modello BDI ed utilizzarli per la creazione degli agenti del sistema `magazzino';
    \item Approfondire la conoscenza del framework JADE ed utilizzarlo come architettura (TODO architettura è corretto?) base per la comunicazione degli agenti distribuiti;
    \item Definire una ontologia rappresentante il magazzino da utilizzare per la definizione di azioni e conoscenze;
    \item Utilizzare un approccio test-driven per monitorare il corretto funzionamento del sistema. (TODO da fare)
\end{itemize}

\subsection{Risultati attesi}
Il risultato atteso è un sistema simulante un magazzino. Questo dovrà accettare ordini entranti e adoperarsi per il loro corretto completamento.\\
Si prevede la presenza di una interfaccia utente tramite la quale sia possibile sottomettere ordini, eseguire task e visualizzare lo stato del sistema. Quest'ultima caratteristica non sarà tuttavia considerata un elemento obbligatorio nella soluzione finale e ci si riserva la possibilità di cancellarne lo sviluppo in caso di problemi o se i tempi di sviluppo dovessero estendersi troppo (TODO).\\
Gli artefatti derivanti dallo sviluppo del progetto saranno immagini docker rappresentanti le entità del sistema ed un software client funzionante su JVM per l'interazione con esso.

% % TODO
% % - detailed description of the project goals, 
% % - requirements,
% % - expected outcomes.
% %
% % Use case Diagrams, examples, or Q/A simulations are welcome.

% L'obiettivo del progetto è lo sviluppo di un sistema di gestione di magazzini basato sull'utilizzo di macchine controllate da agenti autonomi (vedasi Figura \ref{fig:warehouse}). Questo dovrà risultare flessibile per quanto riguarda l'introduzione di nuove macchine e funzionalità e per quanto possibile resistente ai malfunzionamenti. % TODO è resistente? TODO consistenza?
% %
% \begin{figure}[h]\centering
%     \includegraphics[width=.8\textwidth]{section/goals/figure/warehouse.png}
%     \caption{Rappresentazione immaginaria del sistema. Sono visibili le scaffalature, i robot e i punti di raccolta merci.}
%     \label{fig:warehouse}
% \end{figure}
% \\
% Da un punto di vista funzionale, saranno presenti due possibili utenti del sistema: \textit{cliente} e \textit{amministratore}. Questi avranno accesso a differenti tipi di operazioni, le quali possono essere riassunte dal seguente elenco puntato e dal diagramma degli usi visibile in Figura \ref{fig:warehouse-use-case}.
% %
% \begin{itemize}
%     \item piazzare un ordine \textit{[cliente]}
%     \item controllare stato di un ordine \textit{[cliente]}
%     \item supervisionare lo stato del magazzino \textit{[admin]}
%     \item rifornire il magazzino \textit{[admin]}
%     \item sottomettere un task alle macchine del sistema \textit{[admin]}
% \end{itemize}
% %
% \begin{figure}[h]\centering
%     \includegraphics[width=.8\textwidth]{section/goals/figure/warehouse-use_case.png}
%     \caption{Principali azioni effettuabili in base alla tipologia di utente.}
%     \label{fig:warehouse-use-case}
% \end{figure}

% \subsection{Scenarios}
% % TODO
% % Informal description of the ways users are expected to interact with your project.
% % It should describe \emph{how} and \emph{why} a user should use / interact with the system.

% Se i principali 

% \subsection{Self-assessment policy}

% \begin{itemize}
%     \item How should the \emph{quality} of the \emph{produced software} be assessed?
    
%     \item How should the \emph{effectiveness} of the project outcomes be assessed?
% \end{itemize}

%%%%%%%%%%%%%%%%%%%%%%%%%%%%%%%%%%%%%%%%%%%%%%%%%%%%%%%%%%%%%%%%%%%%%%%%%%%%%%%
%   STYLISTIC NOTES
%%%%%%%%%%%%%%%%%%%%%%%%%%%%%%%%%%%%%%%%%%%%%%%%%%%%%%%%%%%%%%%%%%%%%%%%%%%%%%%

\section*{Stylistic Notes}

Use a uniform style, especially when writing formal stuff: $X$, X, $\mathbf{X}$, $\mathcal{X}$, \texttt{X} are all different symbols possibly referring to different entities. 

This is a very short paragraph.

This is a longer paragraph (notice the blank line in the code).
It composed by several sentences.
%
You're invited to use comments within \texttt{.tex} source files to separate sentences composing the same paragraph.

Paragraph should be logically atomic: a subordinate sentence from one paragraph should always refer to another sentence from within the same paragraph.

The first line of a paragraph is usually indented.
%
This is intended: it is the way \LaTeX{} lets the reader know a new paragraph is beginning.

Use the \href{https://en.wikibooks.org/wiki/LaTeX/Source_Code_Listings}{\texttt{listing}} package for inserting scripts into the \LaTeX{} source.

\nocite{*} % Includes all references from the `references.bib` file
\bibliographystyle{plain}
\bibliography{references}

\end{document}
