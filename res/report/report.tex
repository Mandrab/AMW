\documentclass{scrartcl}
\usepackage[utf8]{inputenc}
\usepackage{hyperref}
\usepackage{url}
\usepackage{natbib}
\usepackage{graphicx}
\usepackage{listings}
\usepackage[dvipsnames]{xcolor}

\newcommand{\emailaddr}[1]{\href{mailto:#1}{\texttt{#1}}}

\newcommand{\parag}{\vspace{6pt}\par}

\lstdefinelanguage{Kotlin}{
    basicstyle=\ttfamily\small,
    comment=[l]{//},
    commentstyle={\color{gray}\ttfamily},
    morecomment=[s]{/*}{*/},
    morestring=[b]",
    morestring=[s]{"""*}{*"""},
    morecomment=[s][\color{RedOrange}]{@}{\ }, % annotations
    emph={filter, first, firstOrNull, forEach, lazy, map, mapNotNull, println},
    emphstyle={\color{OrangeRed}},
    identifierstyle=\color{black},
    keywords={!in, !is, abstract, actual, annotation, as, as?, break, by, catch, class, companion, const, constructor, continue, crossinline, data, delegate, do, dynamic, else, enum, expect, external, false, field, file, final, finally, for, fun, get, if, import, in, infix, init, inline, inner, interface, internal, is, lateinit, noinline, null, object, open, operator, out, override, package, param, private, property, protected, public, receiveris, reified, return, return@, sealed, set, setparam, super, suspend, tailrec, this, throw, true, try, typealias, typeof, val, var, vararg, when, where, while},
    keywordstyle={\color{NavyBlue}\bfseries},
    ndkeywords={REQUEST, CONFIRM, INFORM, agent, commandManager, orderManager, robotPicker, warehouseMapper},
    ndkeywordstyle={\color{purple}},
    sensitive=true,
    stringstyle={\color{ForestGreen}\ttfamily},
}

\newlength\Colsep
\setlength\Colsep{10pt}

\title{\LARGE
    AMW - Agents Managed Warehouse
}

\author{
    Paolo Baldini \\ \emailaddr{paolo.baldini6@studio.unibo.it}
}

\begin{document}

\maketitle

%%%%%%%%%%%%%%%%%%%%%%%%%%%%%%%%%%%%%%%%%%%%%%%%%%%%%%%%%%%%%%%%%%%%%%%%%%%%%%%
%   ABSTRACT
%%%%%%%%%%%%%%%%%%%%%%%%%%%%%%%%%%%%%%%%%%%%%%%%%%%%%%%%%%%%%%%%%%%%%%%%%%%%%%%

\begin{abstract}
% TODO
% In the last years warehouses are getting more and more complicated. The increasing need of efficiency has pushed companies to provide their workers with tools that make jobs easier and faster. The largest ones have even substitute part of the human work with automatic machines. In this latter case, the most common approaches try to automatize through use of few PLC that manages the whole system.
% \parag
% Through the implementation of a system in which every machine appear as a robotic agent, I want to give a distributed alternative to the actual centralization practice. In this view, machines should manage incoming orders, retrieve required products and be as flexible and reusable as possible through the presence of an automatic update of \textit{procedural knowledge}.

Rispetto a qualche decennio fa', oggi i magazzini hanno al loro interno una notevole mole di complessità. La ricerca di sempre più efficienza ha infatti portato col tempo le aziende a fornire ai loro dipendenti strumentazioni per rendere più agevole e produttivo il lavoro e, le più grandi di queste, hanno addirittura sostituito parte del lavoro dipendente con macchine totalmente automatizzate. Prendendo in considerazione quest’ultima casistica, la pratica attuale più comune consiste nell’automatizzare il deposito per mezzo di pochi PLC che si occupino della gestione del sistema nel suo complesso.
\parag
Tramite l'implementazione di un sistema in cui ogni macchina appaia come un agente robotico, si vuole fornire un’alternativa distribuita alla pratica di centralizzazione attuale. In questa visione, le macchine dovranno gestire ordini entranti, recuperare i prodotti richiesti e risultare il più flessibili possibile, anche tramite l'upgrade automatico di conoscenza procedurale, nello
svolgimento di attività ”non standard”.
\end{abstract}
\newpage

\tableofcontents
\newpage

%%%%%%%%%%%%%%%%%%%%%%%%%%%%%%%%%%%%%%%%%%%%%%%%%%%%%%%%%%%%%%%%%%%%%%%%%%%%%%%
%   SECTIONS
%%%%%%%%%%%%%%%%%%%%%%%%%%%%%%%%%%%%%%%%%%%%%%%%%%%%%%%%%%%%%%%%%%%%%%%%%%%%%%%

\section{Usage Examples}

Show how to use the produced software artefacts.

Ideally, there should be at least one example for each scenario proposed above.

\section{Usage Examples}

Show how to use the produced software artefacts.

Ideally, there should be at least one example for each scenario proposed above.

\section{Usage Examples}

Show how to use the produced software artefacts.

Ideally, there should be at least one example for each scenario proposed above.

\section{Usage Examples}

Show how to use the produced software artefacts.

Ideally, there should be at least one example for each scenario proposed above.

\section{Usage Examples}

Show how to use the produced software artefacts.

Ideally, there should be at least one example for each scenario proposed above.

\section{Usage Examples}

Show how to use the produced software artefacts.

Ideally, there should be at least one example for each scenario proposed above.

\section{Usage Examples}

Show how to use the produced software artefacts.

Ideally, there should be at least one example for each scenario proposed above.

\section{Usage Examples}

Show how to use the produced software artefacts.

Ideally, there should be at least one example for each scenario proposed above.

%%%%%%%%%%%%%%%%%%%%%%%%%%%%%%%%%%%%%%%%%%%%%%%%%%%%%%%%%%%%%%%%%%%%%%%%%%%%%%%
%   STYLISTIC NOTES
%%%%%%%%%%%%%%%%%%%%%%%%%%%%%%%%%%%%%%%%%%%%%%%%%%%%%%%%%%%%%%%%%%%%%%%%%%%%%%%

\section*{Stylistic Notes}

Use a uniform style, especially when writing formal stuff: $X$, X, $\mathbf{X}$, $\mathcal{X}$, \texttt{X} are all different symbols possibly referring to different entities. 

This is a very short paragraph.

This is a longer paragraph (notice the blank line in the code).
It composed by several sentences.
%
You're invited to use comments within \texttt{.tex} source files to separate sentences composing the same paragraph.

Paragraph should be logically atomic: a subordinate sentence from one paragraph should always refer to another sentence from within the same paragraph.

The first line of a paragraph is usually indented.
%
This is intended: it is the way \LaTeX{} lets the reader know a new paragraph is beginning.

Use the \href{https://en.wikibooks.org/wiki/LaTeX/Source_Code_Listings}{\texttt{listing}} package for inserting scripts into the \LaTeX{} source.

\nocite{*} % Includes all references from the `references.bib` file
\bibliographystyle{plain}
\bibliography{references}

\end{document}
