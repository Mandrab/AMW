\documentclass{scrartcl}
\usepackage[utf8]{inputenc}
\usepackage{hyperref}
\usepackage{url}
\usepackage{natbib}
\usepackage{graphicx}

\newcommand{\emailaddr}[1]{\href{mailto:#1}{\texttt{#1}}}

\newlength\Colsep
\setlength\Colsep{10pt}

\title{\LARGE
    Final Report Template
}

\author{
    Paolo Baldini \\ \emailaddr{paolo.baldini6@studio.unibo.it}
}

\begin{document}

\maketitle

%%%%%%%%%%%%%%%%%%%%%%%%%%%%%%%%%%%%%%%%%%%%%%%%%%%%%%%%%%%%%%%%%%%%%%%%%%%%%%%
%   ABSTRACT
%%%%%%%%%%%%%%%%%%%%%%%%%%%%%%%%%%%%%%%%%%%%%%%%%%%%%%%%%%%%%%%%%%%%%%%%%%%%%%%

\begin{abstract}
In the last years warehouses are getting more and more complicated. The increasing need of efficiency has pushed companies to provide their workers with tools that make jobs easier and faster. The largest ones have even substitute part of the human work with automatic machines. In this latter case, the most common approaches try to automatize through use of few PLC that manages the whole system.
\vspace{6pt}
\par
Through the implementation of a system in which every machine appear as a robotic agent, I want to give a distributed alternative to the actual centralization practice. In this view, machines should manage incoming orders, retrieve required products and be as flexible and reusable as possible through the presence of an automatic update of \textit{procedural knowledge}.
\end{abstract}
\newpage

\tableofcontents
\newpage

%%%%%%%%%%%%%%%%%%%%%%%%%%%%%%%%%%%%%%%%%%%%%%%%%%%%%%%%%%%%%%%%%%%%%%%%%%%%%%%
%   SECTIONS
%%%%%%%%%%%%%%%%%%%%%%%%%%%%%%%%%%%%%%%%%%%%%%%%%%%%%%%%%%%%%%%%%%%%%%%%%%%%%%%

\section{Usage Examples}

Show how to use the produced software artefacts.

Ideally, there should be at least one example for each scenario proposed above.

\section{Usage Examples}

Show how to use the produced software artefacts.

Ideally, there should be at least one example for each scenario proposed above.

\section{Usage Examples}

Show how to use the produced software artefacts.

Ideally, there should be at least one example for each scenario proposed above.

\section{Usage Examples}

Show how to use the produced software artefacts.

Ideally, there should be at least one example for each scenario proposed above.

\section{Usage Examples}

Show how to use the produced software artefacts.

Ideally, there should be at least one example for each scenario proposed above.

\section{Usage Examples}

Show how to use the produced software artefacts.

Ideally, there should be at least one example for each scenario proposed above.

\section{Usage Examples}

Show how to use the produced software artefacts.

Ideally, there should be at least one example for each scenario proposed above.

%%%%%%%%%%%%%%%%%%%%%%%%%%%%%%%%%%%%%%%%%%%%%%%%%%%%%%%%%%%%%%%%%%%%%%%%%%%%%%%
%   STYLISTIC NOTES
%%%%%%%%%%%%%%%%%%%%%%%%%%%%%%%%%%%%%%%%%%%%%%%%%%%%%%%%%%%%%%%%%%%%%%%%%%%%%%%

\section*{Stylistic Notes}

Use a uniform style, especially when writing formal stuff: $X$, X, $\mathbf{X}$, $\mathcal{X}$, \texttt{X} are all different symbols possibly referring to different entities. 

This is a very short paragraph.

This is a longer paragraph (notice the blank line in the code).
It composed by several sentences.
%
You're invited to use comments within \texttt{.tex} source files to separate sentences composing the same paragraph.

Paragraph should be logically atomic: a subordinate sentence from one paragraph should always refer to another sentence from within the same paragraph.

The first line of a paragraph is usually indented.
%
This is intended: it is the way \LaTeX{} lets the reader know a new paragraph is beginning.

Use the \href{https://en.wikibooks.org/wiki/LaTeX/Source_Code_Listings}{\texttt{listing}} package for inserting scripts into the \LaTeX{} source.

\nocite{*} % Includes all references from the `references.bib` file
\bibliographystyle{plain}
\bibliography{references}

\end{document}
